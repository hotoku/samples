\documentclass{jsarticle}

\newcommand{\normal}{\mathrm{N}}
\newcommand{\ampeq}{&=&}
\newcommand{\cov}{\mathrm{Cov}}
\newcommand{\E}{\mathrm{E}}
\newcommand{\iid}{\mathrm{i.i.d.}}


\usepackage{amsmath}	% required for `\matrix' (yatex added)
\usepackage{bm}


\begin{document}

$\epsilon_i, i = 1, 2, \ldots, N \ldots \sim \iid \normal (0, \sigma^2)$という乱数列に対して、
\begin{eqnarray*}
 x_i \ampeq \left\{
             \begin{array}{cc}
              \epsilon_i & i = 1, 2 \\              
              2x_{i-1} - x_{i-2} + \epsilon_i& i > 2 
             \end{array}
            \right.
\end{eqnarray*}
と定義する。

これをもう少し書いてみると
\begin{eqnarray*}
 x_1 \ampeq \epsilon_1 \\
 x_2 \ampeq \epsilon_2 \\
 x_3 \ampeq 2x_2 - x_1 + \epsilon_3 = 2\epsilon_2 - \epsilon_1 + \epsilon_3
  = \epsilon_3 + 2\epsilon_2 - \epsilon_1 \\
 x_4 \ampeq 2x_3 - x_2 + \epsilon_4 = 2(\epsilon_3 + 2\epsilon_2 - \epsilon_1) - \epsilon_2 + \epsilon_4
  = \epsilon_4 + 2\epsilon_3 + 3\epsilon_2 - 2\epsilon_1 \\
 x_5 \ampeq 2x_4 - x_3 + \epsilon_5 = 
  2(\epsilon_4 + 2\epsilon_3 + 3\epsilon_2 - 2\epsilon_1) - 
  (\epsilon_3 + 2\epsilon_2 - \epsilon_1) +
  \epsilon_5 \\
 \ampeq
  \epsilon_5 + 2\epsilon_4 + 3\epsilon_3 + 4\epsilon_2 -3\epsilon_1 \\
 x_6 \ampeq 2x_5 - x_4 + \epsilon_6 = \epsilon_6 + 
  2(\epsilon_5 + 2\epsilon_4 + 3\epsilon_3 + 4\epsilon_2 -3\epsilon_1) - 
  (\epsilon_4 + 2\epsilon_3 + 3\epsilon_2 - 2\epsilon_1) \\
 \ampeq 
  \epsilon_6 + 2\epsilon_5 + 3\epsilon_4 + 4\epsilon_3 + 5\epsilon_2 - 4\epsilon_1
\end{eqnarray*}
となる。行列で書くと
\begin{eqnarray*}
 \left(
  \begin{matrix}
   x_1 \\
   x_2 \\
   x_3 \\
   x_4 \\
   x_5 \\
   x_6 \\
   \dots \\
   x_N
  \end{matrix}
 \right)
 \ampeq
 \left(
  \begin{matrix}
   1 \\
   0 & 1 \\
   -1 & 2 & 1 \\
   -2 & 3 & 2 & 1 \\
   -3 & 4 & 3 & 2 & 1 \\
   -4 & 5 & 4 & 3 & 2 & 1 \\
   \dots \\
   -N+2 & N - 1 & N - 2 & N - 3 & N - 4 & \cdots & 1
  \end{matrix}
 \right)
 \left(
  \begin{matrix}
   \epsilon_1 \\
   \epsilon_2 \\
   \epsilon_3 \\
   \epsilon_4 \\
   \epsilon_5 \\
   \epsilon_6 \\
   \dots \\
   \epsilon_N
  \end{matrix}
 \right) \\
\end{eqnarray*}
となる。これを
\begin{eqnarray*}
 \bm{x} \ampeq A \bm{\epsilon}
\end{eqnarray*}
と書いておく. $\bm{x}$の共分散行列は
\begin{eqnarray*}
 \cov[\bm{x}] = \E[\bm{x}\bm{x}'] = \E[A\bm{\epsilon}\bm{\epsilon}'A'] = \sigma^2 AA'
\end{eqnarray*}
AA'は正定値対称行列なので、直行行列で分解できる
\begin{eqnarray*}
 \cov[\bm{x}] = \sigma^2 AA'= \sigma^2 \Sigma \Lambda \Sigma'
\end{eqnarray*}

新しい乱数列 $\eta_i, i = 1, \ldots, N ~ \mathrm{i.i.d.} \normal(0, \sigma^2)$を考えて、
\begin{eqnarray*}
 \bm{y} \ampeq \Sigma \Lambda^{\frac{1}{2}} \bm{\eta}
\end{eqnarray*}
と定義すると、
\begin{eqnarray*}
 \cov[\bm{y}] = \sigma^2 \Sigma \Lambda^{\frac{1}{2}} \Lambda^{\frac{1}{2}} \Sigma' = \cov[\bm{x}]
\end{eqnarray*}
となる。




\end{document}